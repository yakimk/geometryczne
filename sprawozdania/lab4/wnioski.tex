\quad 
Skuteczność Algorytmu:
Algorytm zamiatania wykazuje się wysoką skutecznością w identyfikowaniu przecięć odcinków na płaszczyźnie. Jego zdolność do obsługi różnych przypadków przecięć, takich jak przecięcia proste, styczne czy nakładające się odcinki, sprawia, że jest on wszechstronnym narzędziem do rozwiązania problemów geometrii obliczeniowej związanego z odcinkami.

Złożoność Obliczeniowa:
Warto zauważyć, że algorytm zamiatania ma relatywnie niską złożoność obliczeniową w porównaniu do niektórych innych metod wyznaczania przecięć odcinków. Jego efektywność staje się szczególnie istotna w przypadku dużej liczby odcinków, gdzie inne podejścia mogą być bardziej kosztowne obliczeniowo.

Przydatność w Praktyce:
Algorytm zamiatania znajduje zastosowanie w wielu praktycznych dziedzinach, takich jak grafika komputerowa, planowanie tras czy analiza obrazów medycznych. Jego prostota implementacyjna i skuteczność czynią go atrakcyjnym narzędziem dla programistów i badaczy zajmujących się problemami geometrii obliczeniowej.

Obsługa Sytuacji Wyjątkowych:
Omawiane sprawozdanie uwzględnia obsługę różnych przypadków szczególnych, takich jak wspólne punkty i brzegi odcinków. To potwierdza, że algorytm zamiatania jest elastyczny i potrafi radzić sobie z różnorodnymi sytuacjami, co zwiększa jego użyteczność w praktyce.

Wyzwania Implementacyjne:
Pomimo ogólnej efektywności, implementacja algorytmu zamiatania może wymagać uwzględnienia wielu szczegółów. W przypadku bardziej skomplikowanych scenariuszy, programiści powinni być świadomi potencjalnych wyzwań związanych z obsługą sytuacji granicznych i optymalizacją kodu.

Jedną ze zmian, którą trzeba było zrealizować było zwracanie w wyniku triangulacji, zarówno oryginalnego 
wielokąta jak i triangulacji po indeksach. To pozwala na wygodny odczyt tej struktury i możliwości szybkiej wizualizacji wyników. Także 
ten sposób przechowywania wyniku algorytmu uniemożliwia błędy związane ze złym dopasowaniem triangulacji do właściwego wielokąta, gdyż zarówno wielokąt, jak i wynik są przechowywane razem. 
Jeszcze jedną rzeczą na którą trzeba było zwrócić uwagę była reprezentacja triangulacji jako listy par indeksów wierzchołków, zamiast listy par współrzędnych wierzchołków. 
To pozwala przede wszystkim na zmiejszenie ilości używanej pamięci, ale również jest bardziej eleganckim rowiązaniem, które upraszcza implementacje algorytmu.