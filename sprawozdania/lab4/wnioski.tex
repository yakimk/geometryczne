\quad 
Wszystkie zaimplementowane procedury i algorytmy zadziałały poprawnie dla testowanych wielokątów. 
Ponadto algorytm poprawnie nie generuje duplikatów krawędzi. 
Użyte struktury danych doskonale pasowały do zadanych problemów - pozwalały na szybki dostęp do potrzebnych elementów, nie wymagały dodatkowych konwersji. 


Jedną ze zmian, którą trzeba było zrealizować było zwracanie w wyniku triangulacji, zarówno oryginalnego 
wielokąta jak i triangulacji po indeksach. To pozwala na wygodny odczyt tej struktury i możliwości szybkiej wizualizacji wyników. Także 
ten sposób przechowywania wyniku algorytmu uniemożliwia błędy związane ze złym dopasowaniem triangulacji do właściwego wielokąta, gdyż zarówno wielokąt, jak i wynik są przechowywane razem. 
Jeszcze jedną rzeczą na którą trzeba było zwrócić uwagę była reprezentacja triangulacji jako listy par indeksów wierzchołków, zamiast listy par współrzędnych wierzchołków. 
To pozwala przede wszystkim na zmiejszenie ilości używanej pamięci, ale również jest bardziej eleganckim rowiązaniem, które upraszcza implementacje algorytmu.