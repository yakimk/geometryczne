\quad Algorytm zamiatania (ang. sweep line algorithm) to efektywna metoda rozwiązania problemu wyznaczania przecięcia się odcinków na płaszczyźnie. Algorytm ten można opisać w kilku krokach:


\begin{enumerate}
    \item 
    \textbf{Sortowanie punktów końcowych odcinków:}
    Algorytm rozpoczyna się od posortowania punktów końcowych odcinków według współrzędnej x. Dzięki temu możemy ustalić kolejność, w jakiej algorytm będzie przetwarzał odcinki.
    
    \item  \textbf{Sweepline:}
Algorytm używa 'miotły', która przesuwa się przez posortowane punkty. Linia ta reprezentuje aktualne położenie algorytmu na płaszczyźnie.

\item  \textbf{Struktura danych do przechowywania aktywnych odcinków:}
Tworzona jest struktura danych, najczęściej używane to drzewo binarne przeszukiwań lub struktura zrównoważona, które przechowuje aktualnie przecinane przez sweepline odcinki.

\item  \textbf{Przeszukiwanie sweepline:}
Sweepline przemieszcza się przez posortowane punkty, a w miarę tego ruchu algorytm aktualizuje strukturę danych, śledząc, które odcinki są obecnie przecinane przez sweepline.

\item  \textbf{Znajdowanie przecięć:}
Przy każdym ruchu sweepline, algorytm sprawdza, czy doszło do przecięcia pomiędzy aktualnym odcinkiem, a innymi odcinkami znajdującymi się nad lub pod sweepline. W przypadku wykrycia przecięcia, jest ono rejestrowane.

\item  \textbf{Obsługa przypadków szczególnych:}
Algorytm uwzględnia różne przypadki przecięć, takie jak przecięcia proste, styczne, czy nakładające się odcinki. Obsługuje również sytuacje szczególne, takie jak wspólne punkty i brzegi odcinków.

\item  \textbf{Aktualizacja struktury danych:}
W miarę poruszania się sweepline, struktura danych jest aktualizowana, a odcinki, które już zostały przetworzone, są usuwane z niej.

\end{enumerate}

Algorytm zamiatania jest stosunkowo prosty do zrozumienia i implementacji, a jego złożoność czasowa wynosi $O(n log n)$, gdzie n to liczba odcinków. Działa efektywnie, zwłaszcza w przypadku dużych zbiorów danych, co czyni go popularnym narzędziem w problemach związanych z geometrią obliczeniową.






