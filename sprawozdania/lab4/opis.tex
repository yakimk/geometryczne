\quad Algorytm zamiatania (ang. sweep line algorithm) to efektywna metoda rozwiązania problemu wyznaczania przecięcia się odcinków na płaszczyźnie. Algorytm ten można opisać w kilku krokach:


\begin{enumerate}
    \item \textbf{Tworzenie listy zdarzeń:} Najpierw dla każdego odcinka dodajemy do listy jego punkt początkowy, oznaczając jako zdarzenie początku odcinka i końcowy, oznaczając go odpowiednio, jako zdarzenia końca odcinka.
    \item 
    \textbf{Sortowanie punktów listy zdarzeń:}
    Sortujemy listę zdarzeń po współrzędnej $x$ punktów do niej należących.
    
    \item  \textbf{Pętla główna:}
    Dla każdego zdarzenia z listy zdarzeń, sprawdzamy czy jest zdarzeniem początkowym, w tym przypadku dodajemy go do Sortedset (zbioru posortowanego). Jeżeli mamy doczynienie z punktem końcowym - usuwamy go z naszego zbioru.

\item  \textbf{Pętla wewnętrzna:}
W przypadku odcinku początkowego, sprawdzamy czy się przecina z którymś z aktywnych odcinków (odcinków ze zbioru posortowanego), jeżeli współrzędna $x$ naszego punktu (zdarzenia) jest większa od współrzędnej $x$ rozpatrywanego zdarzenia oraz te odcinki się przecinają - dodajemy przecięcie do listy przecieć razem z indeksami prostych, które się przecinają.

\item  \textbf{Wynik:}
Zwracamy listę przecięć i indeksów odcinków.
\end{enumerate}

% Algorytm zamiatania jest stosunkowo prosty do zrozumienia i implementacji, a jego złożoność czasowa wynosi $O(n log n)$, gdzie n to liczba odcinków. Działa efektywnie, zwłaszcza w przypadku dużych zbiorów danych, co czyni go popularnym narzędziem w problemach związanych z geometrią obliczeniową.






