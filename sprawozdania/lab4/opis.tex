\quad Algorytm zamiatania (ang. sweep line algorithm) to efektywna metoda rozwiązania problemu wyznaczania przecięcia się odcinków na płaszczyźnie. Algorytm ten można opisać w kilku krokach:


\begin{enumerate}
    \item \textbf{Tworzenie listy zdarzeń:} Każde zdarzenie jest punktem. Zdarzenia przechowujemy w zbiorze posortowanym (Sorted Set), które są posortowane ze względu na swoją współrzędną $x$.
    \item  \textbf{Pętla główna:}
    Po kolei wyciągamy zdarzenia z listy zdarzeń i obsługujemy je za pomocą funkcji handle\_event.

\item  \textbf{Pętla wewnętrzna (handle\_event):}
Najpierw sprawdzamy czy zdarzenie jest zdarzeniem początku lub końca odcinka (odpowiedną informację przechowujemy w słowniku, żeby zapewnić szybki dostęp).

Jeżeli nasze zdarzenie jest zdarzeniem początku odcinka - dodajemy odcinek do zbioru posortowanego z odcinkami, które przecina masza miotła i sprawdzamy czy nie przecina swoich sąsiadów. Jeżeli wykrywamy przecięcie - dodajemy go do zbioru posortowanego wydarzeń.

Jeżeli zdarzenie jest zdarzeniem końca odcinka, usuwamy go z listy odcinków przecinających naszą miotłę.

Jeżeli zdarzenie jest zdarzeniem przecięcia się odcinków - sprawdzamy najpierw czy nie dodaliśmy już przecięcia tej pary odcinków (używamy do tego celu słowniku, w którym przechowujemy pary prostych, przecięcia których już zostały zarejestrowane). 
A pod koniec aktualizujemy (sortujemy) zbiór prostych przecinających miotłę, ze względu na ich współrzędną $y$ po tym przecięciu.
\item  \textbf{Wynik:}
Zwracamy listę przecięć i indeksów odcinków.
\end{enumerate}

\subsection{Struktury zdarzeń i stanu miotły} 

Zarówno struktura zdarzeń, jak i struktura stanu miotły była zaimplementowana za pomocą zbioru posortowanego (Sorted Set) w pythonie. W przypadku miotły przechowujemy w zbiorze odcinki (objekty klasy Line), posortowane ze wzgłedu na współrzędną $y$ w punkcie $x$, który był punktem ostaniego zdarzenia. Zbiór zdarzeń sortujemy po współrzędnej $x$ każdego zdarzenia (zdarzenia w naszym przypadku, to po prostu punkty).

\subsection{Wykrywanie przecięć wielokrotnych} 

Żeby zapobiec sytuacji, w której zapisujemy jedno przecięcie wiele razy, używamy słownkika, do którego zapisujemy pary prostych, przecięcia których jużzostały zarejestrowane. Przykolejnych przecięciach sprawdzamy czy dana para prostych nie wystąpiła już wcześniej. Słownik w pythonie pozwala na szybkie sprawdzenie czy dany klucz już był zapisany do tego słownika, co pozwala nie trtacić na tej operacji złożoności.

\subsection{Wykrywanie przecięć wielokrotnych w algorymtie, który sprawdza czy w zbiorze prostych występują przecięcia}

Nie musimy wykonywać podobnych operacji w przypadku tego algorymtu, gdyz wykrycie jednego przecięcia już powoduje zatrzymanie programu.
% Algorytm zamiatania jest stosunkowo prosty do zrozumienia i implementacji, a jego złożoność czasowa wynosi $O(n log n)$, gdzie n to liczba odcinków. Działa efektywnie, zwłaszcza w przypadku dużych zbiorów danych, co czyni go popularnym narzędziem w problemach związanych z geometrią obliczeniową.






