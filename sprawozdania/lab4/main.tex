\documentclass{sprawozdanie-agh}

\usepackage[utf8]{inputenc}
\usepackage{listings}
\usepackage{xcolor}
\usepackage{graphicx}
\usepackage{caption}
\usepackage{graphicx}
\usepackage{lipsum}
\usepackage{wrapfig}
\usepackage{subcaption}
\usepackage{accsupp}
\usepackage{array}
\usepackage{multirow}
\usepackage{amsmath, siunitx}    
\graphicspath{ {./img/} }
\makeatletter

\begin{document}

\przedmiot{Algorytmy geometryczne}
\tytul{Laboratorium 4}
\podtytul{Przecinanie się odcinków}
\kierunek{Informatyka}
\autor{Kyrylo Iakymenko}
\data{Kraków, 5 grudnia 2023}

\stronatytulowa{}

\section{Wprowadzenie}
\subsection{Cel ćwiczenia}
\quad To ćwiczenie ma na celu zapoznanie się z metodami generacji 
losowych punktów oraz badanie metod klasyfikacji położenia punktów na płaszczyźnie 
względem prostej. 
\subsection{Położenie punktu względem prostej}

\quad Położenie punktu względem prostej będziemy wyznaczać obliczjąc
dane wyznaczniki. Wyznaczniki pozwalają określić położenie
punktu c względem prostej która jest wyznaczona przez punkty a i b.
Jeżeli wyznacznik jest większy od 0 to punkt znajduje się z lewej strony prostej, jeżeli jest mniejszy
od 0  to
punkt znajduje się po prawej stronie prostej, a jeżeli wartość wyznacznika
jest równa 0 (lub jej wartość bezwzględna $< \varepsilon$) to punkt leży na prostej.

\quad Pomimo, że
powyższe wyznaczniki są sobie równoważne to na skutek
niedoskonałości reprezentacji liczb rzeczywistych w komputerze wyniki
mogą się różnić w zależności od użytego wyznacznika.

$$
(1)\det(a, b, c)= \begin{vmatrix}
       a_{x} - c_{x} & a_{y} - c_{y} \\
       b_{x} - c_{x} & b_{y} - c_{y} 
              \end{vmatrix}.\\
              $$
              $$
(2)\det(a, b, c) = \begin{vmatrix}
    a_x & a_y & 1\\
    b_x & b_y & 1\\
    c_x & c_y & 1
\end{vmatrix}.\\
$$

\section{Krótki opis ćwiczenia}
\quad Na potrzeby ćwiczenia stworzyliśmy 6 zbiorów odcinków na płaszczyźnie.
Zostały wybrane odpowiednio, żeby przetestować działanie algorytmu w przypadkach zarówno lososwych jak i ekstremalnych.
\begin{figure}[!h]
    \centering
    \begin{subfigure}{.5\textwidth}
      \centering
      \includegraphics[width=.9\linewidth]{polygon_5.png}
      \caption*{Rys. 1: Pięciokąt.}
      \label{fig:sub1}
    \end{subfigure}%
    \begin{subfigure}{.5\textwidth}
      \centering
      \includegraphics[width=.9\linewidth]{christmas_tree.png}
      \caption*{Rys. 2: Choinka.}
      \label{fig:sub2}
    \end{subfigure}
    \label{fig:test}
    \end{figure}

\newpage

    \begin{figure}[!h]
    \centering
    \begin{minipage}{.5\textwidth}
      \centering
      \includegraphics[width=.9\linewidth]{polygon_weird.png}
      \caption*{Rys. 3: Wielokąt 1.}
      \label{fig:test1}
    \end{minipage}%
    \begin{minipage}{.5\textwidth}
      \centering
      \includegraphics[width=.9\linewidth]{polygon_weird_2.png}
      \caption*{Rys. 4: Wielokąt 2.}
      \label{fig:test2}
    \end{minipage}
    \end{figure}
    \begin{figure}[!h]
        \centering
        \begin{minipage}{.5\textwidth}
          \centering
          \includegraphics[width=.9\linewidth]{polygon_weird_3.png}
          \caption*{Rys. 5: Wielokąt 3.}
          \label{fig:test1}
        \end{minipage}%
        \begin{minipage}{.5\textwidth}
          \centering
          \includegraphics[width=.9\linewidth]{polygon_50.png}
          \caption*{Rys. 6: 50-kąt foremny.}
          \label{fig:test2}
        \end{minipage}
        \end{figure}
\section{Algorytmy}
\subsection{Algorytm triangulacji}
\quad Triangulacja wielokąta monotonicznego to proces podziału wielokąta na trójkąty, zachowując przy tym monotoniczność. Wielokąt $y$-monotoniczny to taki, którego prosta pozioma przecina go maksymalnie dwukrotnie. Algorytm triangulacji wielokąta monotonicznego można opisać w kilku krokach:

\begin{enumerate}
\item \textbf{Sortuj wierzchołki:} Posortuj malejąco wierzchołki wielokąta względem ich współrzędnej $y$. Dodaj dwa pierwsze wierzchołki do stosu.

\item \textbf{Podziel wierzchołki na lewy i prawy łańcuchy:} Za pomocą algorytmu działającego w $O(n)$ dzielimy wierzchołki na należące do prawego ido lewego łancuchów. 
Wyniki przechowujemy w słowniku.

\item \textbf{Główna pętla:} Dla każdego z kolenych wierzchołków od $i = 3$ do $i = n-1$ sprawdzamy czy jest na tym samym łancuchu co poprzedni

\begin{itemize}
    \item jeżeli tak, 
    dodajemy nasz punkt na stos
    sprawdzamy za pomocą wyznacznika czy prosta pomiędzy pierwszym a trzecim wierzchołkami na stosie zawiera się w wielokącie, jeśli tak to tworzymy prostą między nimi, następnie usuwamy drugi wierzchołek na stosie.
    
    
    \item jeżeli nie, 
    ususwamy pierwszy wierzchołek $p$ ze stosu,
    tworzymy prostą między nim a wszystkimi wierzchołkami na stosie oprócz ostatniego, a następnie dodajemy na stos badany wierzchołek oraz wierzchołek poprzedzający $p$.
\end{itemize}

\item \textbf{Połacz wierzchołki na stosie: } W kroku ostatnim łączymy wierzchołki, które pozostały na stosie, oprócz pierwszego i ostatniego, z wierzchołkiem o najmniejszej wspólrzędnej $y$.
\item \textbf{Zwróć listę połaczonych par wierżchołków}

\end{enumerate}


Przedstawiony algorytm jest efektywny i działa w czasie \(O(n \log n)\), gdzie \(n\) to liczba wierzchołków wielokąta. Triangulacja wielokąta monotonicznego jest często stosowana w grafice komputerowej i w problemach geometrii obliczeniowej.

Struktura przechowująca wielokąt opiera się na posortowanej liście wierzchołków oraz słownika, za pomocą którego możemy szybko sprawdzić do którego z łancuchów należy rozpatrywany wierzchołek. 
Jest to potrzebne przy sprawdzaniu czy linia między punktami jest wewnątrz wielokąta.
\subsection{Algorytm sprawadzenia $y$-monotoniczności wielokąta}
\quad Nasz algorytm najpierw dzieli wierzchołki na 5 rodzajów: 
\begin{enumerate}
    \item Początkowy gdy obaj jego sąsiedzi leżą
    poniżej i kąt wewnętrzny $< \pi$.
    \item Końcowy, gdy obaj jego sąsiedzi leżą
    powyżej i kąt wewnętrzny $< \pi$.
    \item Łączący, gdy obaj jego sąsiedzi leżą
    powyżej i kąt wewnętrzny $> \pi$.
    \item Dzielący, gdy obaj jego sąsiedzi leżą
    poniżej i kąt wewnętrzny $> \pi$.
    \item prawidłowy, gdy ma jednego sąsiada powyżej, drugiego - poniżej.
\end{enumerate}

Robi to licząc wyznacznik i odpowiednio porównując współrzędne swoich sąsiadów. 
Wtedy sprawdzenie czy wielokąt jest $y-$monotoniczny sprowadza się do sprawdzenia, czy nasz 
wielokąt posiada wierzchołki łączące lub dzielące. Jeśli tak - nie jest monotoniczny, w przeciwnym wypadku - jest.
\newpage
\section{Wyniki triangulacji}
\begin{figure}[!h]
    \centering
    \begin{subfigure}{.5\textwidth}
      \centering
      \includegraphics[width=.9\linewidth]{0_int.png}
      \caption*{Rys. 7}
      \label{fig:sub1}
    \end{subfigure}%
    \begin{subfigure}{.5\textwidth}
      \centering
      \includegraphics[width=.9\linewidth]{1_int.png}
      \caption*{Rys. 8}
      \label{fig:sub2}
    \end{subfigure}
    \label{fig:test}
    \end{figure}

% \newpage

    \begin{figure}[!h]
    \centering
    \begin{minipage}{.5\textwidth}
      \centering
      \includegraphics[width=.9\linewidth]{2_int.png}
      \caption*{Rys. 9}
      \label{fig:test1}
    \end{minipage}%
    \begin{minipage}{.5\textwidth}
      \centering
      \includegraphics[width=.9\linewidth]{3_int.png}
      \caption*{Rys. 10}
      \label{fig:test2}
    \end{minipage}
    \end{figure}
    \begin{figure}[!h]
        \centering
        \begin{minipage}{.5\textwidth}
          \centering
          \includegraphics[width=.9\linewidth]{4_int.png}
          \caption*{Rys. 11}
          \label{fig:test1}
        \end{minipage}%
        \begin{minipage}{.5\textwidth}
          \centering
          \includegraphics[width=.9\linewidth]{5_int.png}
          \caption*{Rys. 12}
          \label{fig:test2}
        \end{minipage}
        \end{figure}
\section{Podsumowanie}
\quad Podczas implementacji algorytmu zamiatania przetestowano go na różnych zestawach danych, a uzyskane wyniki i wnioski zostały szczegółowo przedstawione w sprawozdaniu. Wnioski te obejmują skuteczność algorytmu w różnych scenariuszach. Największym wyzwaniem była poprawna obsługa wykrytych przecięc.

W wyniku nasz algorytm poprawnie wykrywa przecięcia odcinków, nie zapisuje duplikatów przeciec oraz ma złożoność $O(nlogn)$.

Warto zaznczyć, że nasz algorytm może być modyfikowany porzez użycie innych struktur zdarzeń i miotły i sam algorytm zamiatania jest tylko wzorcem, na podstawie, którego można zaimplementować wiele różnych, ale dających ten sam wynik algorytmów.


\end{document}