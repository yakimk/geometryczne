\quad Porównując ilość zaklasyfikowanych punktów dla różnych $\varepsilon$ dostaliśmy róźne wyiniki 
dla róznych metod obliczenia wyznacznika. 
Klasyfikacja punktów z podpunktów a oraz c dla większości
przypadków dawała identyczne wyniki niezależnie od użytego
wyznacznika czy epsilona (tolerancji). Spowodowane jest to tym, że
istnieje bardzo małe prawdopodobieństwo, iż punkt będzie znajdował się
dokładnie na prostej, z dokładnością do użytej tolerancji. W podpunkcie
b wybór epsilona nie miał znaczenia jednak wybór wyznacznika już tak.
Jest to spowodowane
najprawdopodobniej tym, że współrzędne tych punktów mają duże
wartości bezwzględne. Operacja klasyfikacji punktów wymaga jednak
dużej precyzji, która jest tracona z powodu dużych wartości które są
generowane. Z tego możemy dojść do wniosku, że użycie wyznaczników $3 \times 3$ daje 
nam większą precyzje w obliczeniach, a też patrząc na wyniki porównania wydajności funkcji 
liczących wyznacznik, są wykonywane tak samo szybko, jak i funkcje liczenia wyznacznika $2 \times 2$.
\par
\quad Patrząc na porównanie precyzji float64 z float32, możemy dojść do wniosku, 
że float64 daje (jak wypadało się spodziewać) dużo większą precyzje obliczeń. 
Widoczna jest poważna różnica w obliczeniach już dla $\varepsilon = 10^{-12}$. 
Na tej podstawie na potrzeby algorytmów geometrycznych zalecane jest 
używanie float64.
\par
\quad We wszystkich testach i porównaniach statystycznych dostaliśmy 
wyniki, które zgadzały się z wartościami teoretycznymi. Na tej 
podstawie możemy dojść do wniosku, że w obliczeniach nie było poważnych 
błędów i ćwiczenie było wykonane poprawnie.
