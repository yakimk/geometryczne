\quad Porównując ilość zaklasyfikowanych punktów dla różnych $\varepsilon$ dostaliśmy róźne wyiniki 
dla róznych metod obliczenia wyznacznika. 
Klasyfikacja punktów ze zbiorów a oraz c dawała podobne wyniki niezależnie od użytej
funkcji liczenia wyznacznika czy $\varepsilon$. Spowodowane jest to tym, że
istnieje bardzo małe prawdopodobieństwo, iż punkt będzie znajdował się
dokładnie na prostej, z dokładnością do użytej tolerancji. W podpunkcie
b wybór epsilona nie miał znaczenia jednak wybór wyznacznika już tak.
Jest to spowodowane dużymi odległościami pomiędzy punktami. Operacja klasyfikacji punktów wymaga jednak
dużej precyzji, która jest tracona z powodu dużych wartości które są
generowane. Z tego możemy dojść do wniosku, że użycie wyznaczników $3 \times 3$ daje 
nam większą precyzje w obliczeniach, a też patrząc na wyniki porównania wydajności funkcji 
liczących wyznacznik, są wykonywane tak samo szybko, jak i funkcje liczenia wyznacznika $2 \times 2$.
\par
\quad We wszystkich testach i porównaniach statystycznych dostaliśmy 
wyniki, które zgadzały się z wartościami teoretycznymi. Na tej 
podstawie możemy dojść do wniosku, że w obliczeniach nie było poważnych 
błędów i ćwiczenie było wykonane poprawnie.
