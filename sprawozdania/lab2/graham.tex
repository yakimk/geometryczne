\subsection{Algorytm Grahama}

\quad Algorytm Grahama jest popularnym algorytmem do wyznaczania otoczki wypukłej dla zbioru punktów w przestrzeni dwuwymiarowej. Ten algorytm jest bardziej efektywny niż algorytm Jarvisa i ma złożoność czasową wynoszącą $O(nlog(n))$, gdzie $n$ to liczba punktów w zbiorze. Algorytm Grahama opiera się na wykorzystaniu sortowania punktów względem ich kąta względem punktu startowego.

Poniżej przedstawiam kroki algorytmu Grahama:

Wybieramy punkt startowy, który będzie częścią otoczki wypukłej. W praktyce często wybiera się punkt o najniższej wartości współrzędnej y (jeśli istnieje więcej niż jeden taki punkt, wybiera się ten z najniższą wartością współrzędnej x).

Sortujemy pozostałe punkty ze zbioru według kąta, jaki tworzą względem punktu startowego. Dzięki temu punkty są rozmieszczone w kolejności od najmniejszego kąta do największego kąta. W przypadku, gdy wiele punktów ma ten sam kąt, sortowane są względem odległości od punktu startowego.

Algorytm przetwarza punkty w kolejności rosnących kątów. Początkowo dodajemy punkt startowy do otoczki wypukłej.

Dla każdego punktu, który przetwarzamy, sprawdzamy, czy tworzy on odcinek wypukły w stosunku do ostatnich dwóch punktów na otoczce. Jeśli tak, dodajemy ten punkt do otoczki wypukłej.

Jeśli punkt nie tworzy odcinka wypukłego, to usuwamy ostatni punkt otoczki i sprawdzamy punkt obecny ponownie w kontekście poprzedniego punktu na otoczce.

Algorytm kończy działanie, gdy wszystkie punkty zostały przetworzone, a otoczka wypukła została zakończona.





