\quad Na potrzeby ćwiczenia wygenerujemy $4$ zbiory punktów losowych.
\begin{enumerate}[a)]
    \item Losowe punkty $(x, y)$ w przestrzeni $\mathbb{R}^2$, gdzie $(x, y) \in \left[-10^4,10^4\right]^{2}$.
    \item Losowe punkty $(x, y)$ w przestrzeni $\mathbb{R}^2$, położone na okręgu o promieniu $R = 10^4$.
    \item Losowe punkty w przestrzeni $\mathbb{R}^2$, 
    położone na prostokącie o wierzcholku w $(-10^4, -10^4)$ i $(10^4, 10^4)$.
    \item Losowe punkty w przestrzeni $\mathbb{R}^2$, 
    położone na dolnym i lewym bokack kwadratu o wierzcholku w $(-10^4, -10^4)$ i $(10^4, 10^4)$ i jego przekątnych.
\end{enumerate}
