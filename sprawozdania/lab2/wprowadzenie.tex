\quad Otoczka wypukła stanowi fundamentalne pojęcie w dziedzinie algorytmów geometrycznych oraz obliczeń geometrycznych. Jest to zbiór punktów, które tworzą najmniejszy wielokąt wypukły, który obejmuje wszystkie punkty z danego zbioru. W praktyce znajduje ona zastosowanie w wielu dziedzinach, takich jak grafika komputerowa, przetwarzanie obrazów, planowanie trasy, a nawet w optymalizacji geometrii konstrukcji.

W niniejszym sprawozdaniu skupimy się na omówieniu dwóch popularnych algorytmów służących do wyznaczania otoczki wypukłej: algorytmu Grahama oraz algorytmu Jarvisa. Obie metody pozwalają na efektywne rozwiązanie tego problemu i zostały szeroko wykorzystane w praktyce.

