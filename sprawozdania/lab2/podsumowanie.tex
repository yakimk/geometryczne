W trakcie prowadzenia tego badania przeprowadziliśmy analizę i porównanie dwóch popularnych algorytmów służących do wyznaczania otoczki wypukłej, tj. algorytmów Grahama i Jarvisa. Naszym celem było zrozumienie działania tych algorytmów oraz porównanie ich wydajności przy użyciu różnych zbiorów punktów. Poniżej przedstawiamy główne wnioski z naszych badań:

Algorytm Grahama:

Algorytm Grahama charakteryzuje się złożonością czasową wynoszącą O(n*log(n)), co sprawia, że jest bardziej wydajny niż algorytm Jarvisa.
Jest efektywny w przypadku zbiorów punktów o większej liczbie elementów, co sprawia, że jest bardziej przydatny w rzeczywistych zastosowaniach.
Algorytm Jarvisa:

Algorytm Jarvisa jest prosty w implementacji, ale jego złożoność czasowa wynosi O(nh), gdzie "n" to liczba punktów, a "h" to liczba punktów na otoczce wypukłej.
Jego wydajność maleje znacząco w przypadku zbiorów punktów o dużym rozmiarze, co sprawia, że może być mniej praktyczny w niektórych sytuacjach.
Porównanie:

W naszych badaniach zaobserwowaliśmy, że algorytm Grahama osiąga lepsze wyniki wydajnościowe niż algorytm Jarvisa, zwłaszcza w przypadku dużych zbiorów punktów.
Algorytm Grahama jest bardziej efektywny dzięki zastosowaniu sortowania punktów względem kąta względem punktu startowego, co pozwala na znaczne przyspieszenie procesu wyznaczania otoczki wypukłej.
Podsumowując, zarówno algorytm Grahama, jak i algorytm Jarvisa znajdują zastosowanie w wyznaczaniu otoczki wypukłej, ale wybór między nimi zależy od konkretnych potrzeb i rozmiaru zbioru punktów. Algorytm Grahama jest bardziej efektywny i zwykle bardziej polecany w praktyce, zwłaszcza w przypadku dużych zbiorów punktów, gdzie złożoność czasowa odgrywa kluczową rolę. Jednak algorytm Jarvisa pozostaje przydatnym narzędziem, zwłaszcza w celach edukacyjnych i w przypadku mniejszych zbiorów punktów, gdzie prostota implementacji może mieć znaczenie.




