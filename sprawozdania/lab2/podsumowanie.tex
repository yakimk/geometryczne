\quad W trakcie prowadzenia tego badania przeprowadzona została analiza i porównanie dwóch popularnych algorytmów służących do wyznaczania otoczki wypukłej, tj. algorytmów Grahama i Jarvisa. Celem było zrozumienie działania tych algorytmów oraz porównanie ich wydajności przy użyciu różnych zbiorów punktów. 
Poniżej przedstawione są główne wnioski z  badań.
\par
Algorytm Grahama charakteryzuje się złożonością czasową wynoszącą $O(nlog(n))$, co sprawia, że jest bardziej wydajny niż algorytm Jarvisa dla większości zbiorów losowych, azckolwiek są 
wyjątki co widać po zbiorach $c$ i $d$. W związku ze swoją lepszą złożonością używanie algorytma Grahama jest zalecane dla większości przypadków.

\par
Algorytm Jarvisa jest prosty w implementacji, ale jego złożoność czasowa wynosi $O(nh)$, gdzie $n$ to liczba punktów, a $h$ to liczba punktów na otoczce wypukłej.
Jego wydajność maleje znacząco w przypadku zbiorów z otoczką wypukłą o dużym rozmiarze, co sprawia, że może być mniej praktyczny w niektórych sytuacjach.
Wykorzystanie algorytmu Jarvisa może być zalecane tylko w przypadkach gdy wiemy, że otoczka wypukła składa się ze stosunkowo niedużej ilosci punktów.

Zaobserwowaliśmy, że algorytm Grahama osiąga lepsze wyniki wydajnościowe niż algorytm Jarvisa w przypadku dużych zbiorów punktów losowych. 
Natomiast jeżeli nasz zbiór punktów ma pewną strukturę, która gwarantuje stosunkowo małą otoczkę wypukłą, to użycie algorytmu Jarvisa może być troche lepsze dla przypadków o dużej ilości punktów.
Algorytm Grahama jest bardziej efektywny dzięki zastosowaniu sortowania punktów względem kąta względem punktu startowego, co pozwala na znaczne przyspieszenie procesu wyznaczania otoczki wypukłej.

Podsumowując, zarówno algorytm Grahama, jak i algorytm Jarvisa znajdują zastosowanie w wyznaczaniu otoczki wypukłej, ale wybór między nimi zależy od konkretnych potrzeb i rozmiaru zbioru punktów. Algorytm Grahama jest bardziej efektywny i zwykle bardziej polecany w praktyce, zwłaszcza w przypadku dużych zbiorów punktów, gdzie złożoność czasowa odgrywa kluczową rolę. Jednak algorytm Jarvisa pozostaje przydatnym narzędziem, zwłaszcza w celach edukacyjnych i w przypadku mniejszych zbiorów punktów, gdzie prostota implementacji może mieć znaczenie.