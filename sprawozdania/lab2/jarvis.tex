\subsection{Algorytm Jarvisa}
\quad Algorytm Jarvisa, znany również jako algorytm "Gift Wrapping", to jedna metod wyznaczania otoczki wypukłej dla zbioru punktów w przestrzeni dwuwymiarowej. Ten algorytm jest stosunkowo prosty w implementacji, choć ma złożoność czasową wynoszącą $O(nh)$, gdzie $n$ to liczba punktów w zbiorze, a $h$ to liczba punktów na otoczce wypukłej.

Poniżej przedstawiam kroki algorytmu Jarvisa:

Na początek należy wybrać punkt startowy, który będzie częścią otoczki wypukłej. Często wybiera się punkt o najniższej wartości współrzędnej y (jeśli istnieje więcej niż jeden taki punkt, wybiera się ten z najniższą wartością współrzędnej x).

Ustalamy punkt startowy jako pierwszy punkt otoczki wypukłej.

W kolejnej fazie algorytmu próbujemy znaleźć punkt z otoczki, który znajduje się na prawo od odcinka łączącego bieżący punkt otoczki z poprzednim punktem otoczki. To oznacza, że nowy punkt musi tworzyć odcinek wypukły w stosunku do bieżącej otoczki.

Algorytm działa w pętli, w której znajduje się następny punkt otoczki, który spełnia kryterium odcinka wypukłego, aż wróci do punktu początkowego, co oznacza, że otoczka została zamknięta.

Po znalezieniu kolejnego punktu, dodajemy go do otoczki wypukłej.

Algorytm kończy działanie, gdy wróci do punktu początkowego, tworząc zamkniętą otoczkę wypukłą.