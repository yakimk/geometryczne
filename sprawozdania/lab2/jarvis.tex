\subsection{Algorytm Jarvisa}
Algorytm Jarvisa, znany również jako algorytm "Zawijania prezentu" (ang. "Gift Wrapping"), to jedna z popularnych metod wyznaczania otoczki wypukłej dla zbioru punktów w przestrzeni dwuwymiarowej. Ten algorytm jest stosunkowo prosty w implementacji, choć ma złożoność czasową wynoszącą O(nh), gdzie "n" to liczba punktów w zbiorze, a "h" to liczba punktów na otoczce wypukłej.

Poniżej przedstawiam kroki algorytmu Jarvisa:

Wybierz punkt startowy: Na początek należy wybrać punkt startowy, który będzie częścią otoczki wypukłej. To może być dowolny punkt z dostępnego zbioru punktów, ale często wybiera się punkt o najniższej wartości współrzędnej y (jeśli istnieje więcej niż jeden taki punkt, wybiera się ten z najniższą wartością współrzędnej x).

Inicjalizacja otoczki wypukłej: Ustalamy punkt startowy jako pierwszy punkt otoczki wypukłej.

Znajdź kolejny punkt: W tej fazie algorytmu próbujemy znaleźć punkt z otoczki, który znajduje się na prawo od odcinka łączącego bieżący punkt otoczki z poprzednim punktem otoczki. To oznacza, że nowy punkt musi tworzyć odcinek wypukły w stosunku do bieżącej otoczki.

Główna pętla: Algorytm działa w pętli, w której znajduje się następny punkt otoczki, który spełnia kryterium odcinka wypukłego, aż wróci do punktu początkowego, co oznacza, że otoczka została zamknięta.

Aktualizacja otoczki: Po znalezieniu kolejnego punktu, dodajemy go do otoczki wypukłej.

Zakończenie: Algorytm kończy działanie, gdy wróci do punktu początkowego, tworząc zamkniętą otoczkę wypukłą.

Algorytm Jarvisa jest stosunkowo prosty, ale jego głównym ograniczeniem jest jego wysoka złożoność czasowa, co sprawia, że nie jest najbardziej wydajnym algorytmem w przypadku dużych zbiorów punktów. W praktyce bardziej efektywne są bardziej zaawansowane algorytmy, takie jak algorytm Grahama lub Quickhull. Jednak algorytm Jarvisa nadal znajduje zastosowanie w edukacji oraz w sytuacjach, gdy prostota implementacji jest ważniejsza niż wydajność.




