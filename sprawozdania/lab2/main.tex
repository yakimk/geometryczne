\documentclass{sprawozdanie-agh}

\usepackage[utf8]{inputenc}
\usepackage{listings}
\usepackage{tikz}
\usepackage{xcolor}
\usepackage{graphicx}
\usepackage{caption}
\usepackage{graphicx}
\usepackage{lipsum}
\usepackage{wrapfig}
\usepackage{subcaption}
\usepackage{accsupp}
\usepackage{array}
\usepackage{multirow}
\usepackage{amsmath, siunitx}
\usepackage{booktabs}
\graphicspath{ {./images/} }

\usetikzlibrary{angles,quotes}
\makeatletter

\begin{document}

\przedmiot{Algorytmy geometryczne}
\tytul{Laboratorium 2}
\podtytul{Otoczka wypukła}
\kierunek{Informatyka}
\autor{Kyrylo Iakymenko}
\data{Kraków, 7 listopada 2023}

\stronatytulowa{}

\section{Wprowadzenie}
\subsection{Tabele}
\quad Czasy wykonania algorytmów na zbiorach testowych w zależności od ilości punktów podane w sekundach $[s]$.
\renewcommand{\arraystretch}{2}
\begin{table}[!ht]
    \centering
\begin{tabular}{l  c|c|c|c|}
    & \multicolumn{4}{c}{\textbf{Liczba punktów}} \\ \cline{2-5}     
    \multicolumn{1}{l|}{\textbf{Algorytm}} & $10^3$& $10^4$& $10^5$& $10^6$ \\
   \hline
   \hline
   \multicolumn{1}{l|}{\textbf{Graham}} & 0.007& 0.088& 1.222& 15.013 \\
   \hline
   \multicolumn{1}{l|}{\textbf{Jarvis}} & 0.010& 0.107& 1.771& 25.266 \\
   \cline{2-5}
\end{tabular}
\caption*{Tabela 1: Czasy wykonania algorytmów na zbiorze testowym $a$.}
\end{table}


\begin{table}[!ht]
    \centering
\begin{tabular}{l  c|c|c|}
    & \multicolumn{3}{c}{\textbf{Liczba punktów}} \\ \cline{2-4}     
    \multicolumn{1}{l|}{\textbf{Algorytm}} & 100& $10^3$& $10^4$ \\
   \hline
   \hline
   \multicolumn{1}{l|}{\textbf{Graham}} & 0.001& 0.011& 0.130 \\
   \hline
   \multicolumn{1}{l|}{\textbf{Jarvis}} & 0.008& 0.678& 70.709 \\
   \cline{2-4}
\end{tabular}
\caption*{Tabela 2: Czasy wykonania algorytmów na zbiorze testowym $b$.}
\end{table}

\newpage
\begin{table}[!ht]
    \centering
\begin{tabular}{l  c|c|c|c|c|}
    & \multicolumn{5}{c}{\textbf{Liczba punktów}} \\ \cline{2-6}     
    \multicolumn{1}{l|}{\textbf{Algorytm}} & 100& $10^3$& $10^4$& $10^5$& $10^6$ \\
   \hline
   \hline
   \multicolumn{1}{l|}{\textbf{Graham}} & 0.005& 0.064& 0.739& 9.177& 113.494 \\
   \hline
   \multicolumn{1}{l|}{\textbf{Jarvis}} & 0.001& 0.009& 0.115& 1.242& 13.464 \\
   \cline{2-6}
\end{tabular}
\caption*{Tabela 3: Czasy wykonania algorytmów na zbiorze testowym $c$.}
\end{table}


\begin{table}[!ht]
    \centering
\begin{tabular}{l  c|c|c|c|c|}
    & \multicolumn{5}{c}{\textbf{Liczba punktów}} \\ \cline{2-6}     
    \multicolumn{1}{l|}{\textbf{Algorytm}} & 100& $10^3$& $10^4$& $10^5$& $10^6$ \\
   \hline
   \hline
   \multicolumn{1}{l|}{\textbf{Graham}} & 0.002& 0.027& 0.212& 2.816& 37.919 \\
   \hline
   \multicolumn{1}{l|}{\textbf{Jarvis}} & 0.001& 0.011& 0.077& 0.921& 9.709 \\
   \cline{2-6}
\end{tabular}
\caption*{Tabela 4: Czasy wykonania algorytmów na zbiorze testowym $d$.}
\end{table}

\section{Opis stanowiska}

\section{Przebieg doświadczenia}


\section{Wyniki pomiarów}


\section{Opracowanie wyników}
\section{Podsumowanie}
\end{document}
