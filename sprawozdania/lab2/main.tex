\documentclass{sprawozdanie-agh}

\usepackage[utf8]{inputenc}
\usepackage{listings}
\usepackage{xcolor}
\usepackage{graphicx}
\usepackage{caption}
\usepackage{graphicx}
\usepackage{wrapfig}
\usepackage{subcaption}
\usepackage{accsupp}
\usepackage{array}
\usepackage{amsfonts}
\usepackage[shortlabels]{enumitem}
\usepackage{amsmath, xparse}
\usepackage{listings}
\usepackage{booktabs}

\graphicspath{ {./img/} }

\makeatletter

\begin{document}

\przedmiot{Algorytmy geometryczne}
\tytul{Laboratorium 2}
\podtytul{Otoczka wypukła}
\kierunek{Informatyka}
\autor{Kyrylo Iakymenko}
\data{Kraków, 7 listopada 2023}

\stronatytulowa{}

\section{Wprowadzenie}
\subsection{Cel ćwiczenia}
\quad To ćwiczenie ma na celu zapoznanie się z metodami generacji 
losowych punktów oraz badanie metod klasyfikacji położenia punktów na płaszczyźnie 
względem prostej. 
\subsection{Położenie punktu względem prostej}

\quad Położenie punktu względem prostej będziemy wyznaczać obliczjąc
dane wyznaczniki. Wyznaczniki pozwalają określić położenie
punktu c względem prostej która jest wyznaczona przez punkty a i b.
Jeżeli wyznacznik jest większy od 0 to punkt znajduje się z lewej strony prostej, jeżeli jest mniejszy
od 0  to
punkt znajduje się po prawej stronie prostej, a jeżeli wartość wyznacznika
jest równa 0 (lub jej wartość bezwzględna $< \varepsilon$) to punkt leży na prostej.

\quad Pomimo, że
powyższe wyznaczniki są sobie równoważne to na skutek
niedoskonałości reprezentacji liczb rzeczywistych w komputerze wyniki
mogą się różnić w zależności od użytego wyznacznika.

$$
(1)\det(a, b, c)= \begin{vmatrix}
       a_{x} - c_{x} & a_{y} - c_{y} \\
       b_{x} - c_{x} & b_{y} - c_{y} 
              \end{vmatrix}.\\
              $$
              $$
(2)\det(a, b, c) = \begin{vmatrix}
    a_x & a_y & 1\\
    b_x & b_y & 1\\
    c_x & c_y & 1
\end{vmatrix}.\\
$$
\section{Opis wykorzystanych algorytmów}
\subsection{Algorytm Grahama}

\quad Algorytm Grahama jest popularnym algorytmem do wyznaczania otoczki wypukłej dla zbioru punktów w przestrzeni dwuwymiarowej. Ten algorytm jest bardziej efektywny niż algorytm Jarvisa i ma złożoność czasową wynoszącą $O(nlog(n))$, gdzie $n$ to liczba punktów w zbiorze. Algorytm Grahama opiera się na wykorzystaniu sortowania punktów względem ich kąta względem punktu startowego.

Poniżej przedstawiam kroki algorytmu Grahama:

Wybieramy punkt startowy, który będzie częścią otoczki wypukłej. W praktyce często wybiera się punkt o najniższej wartości współrzędnej y (jeśli istnieje więcej niż jeden taki punkt, wybiera się ten z najniższą wartością współrzędnej x).

Sortujemy pozostałe punkty ze zbioru według kąta, jaki tworzą względem punktu startowego. Dzięki temu punkty są rozmieszczone w kolejności od najmniejszego kąta do największego kąta. W przypadku, gdy wiele punktów ma ten sam kąt, sortowane są względem odległości od punktu startowego.

Algorytm przetwarza punkty w kolejności rosnących kątów. Początkowo dodajemy punkt startowy do otoczki wypukłej.

Dla każdego punktu, który przetwarzamy, sprawdzamy, czy tworzy on odcinek wypukły w stosunku do ostatnich dwóch punktów na otoczce. Jeśli tak, dodajemy ten punkt do otoczki wypukłej.

Jeśli punkt nie tworzy odcinka wypukłego, to usuwamy ostatni punkt otoczki i sprawdzamy punkt obecny ponownie w kontekście poprzedniego punktu na otoczce.

Algorytm kończy działanie, gdy wszystkie punkty zostały przetworzone, a otoczka wypukła została zakończona.






\subsection{Algorytm Jarvisa}
\quad Algorytm Jarvisa, znany również jako algorytm "Gift Wrapping", to jedna metod wyznaczania otoczki wypukłej dla zbioru punktów w przestrzeni dwuwymiarowej. Ten algorytm jest stosunkowo prosty w implementacji, choć ma złożoność czasową wynoszącą $O(nh)$, gdzie $n$ to liczba punktów w zbiorze, a $h$ to liczba punktów na otoczce wypukłej.

Poniżej przedstawiam kroki algorytmu Jarvisa:

Na początek należy wybrać punkt startowy, który będzie częścią otoczki wypukłej. Często wybiera się punkt o najniższej wartości współrzędnej y (jeśli istnieje więcej niż jeden taki punkt, wybiera się ten z najniższą wartością współrzędnej x).

Ustalamy punkt startowy jako pierwszy punkt otoczki wypukłej.

W kolejnej fazie algorytmu próbujemy znaleźć punkt z otoczki, który znajduje się na prawo od odcinka łączącego bieżący punkt otoczki z poprzednim punktem otoczki. To oznacza, że nowy punkt musi tworzyć odcinek wypukły w stosunku do bieżącej otoczki.

Algorytm działa w pętli, w której znajduje się następny punkt otoczki, który spełnia kryterium odcinka wypukłego, aż wróci do punktu początkowego, co oznacza, że otoczka została zamknięta.

Po znalezieniu kolejnego punktu, dodajemy go do otoczki wypukłej.

Algorytm kończy działanie, gdy wróci do punktu początkowego, tworząc zamkniętą otoczkę wypukłą.

\section{Generowanie zbiorów punktów}
\subsection{Opis zbiorów testowych}
\quad Na potrzeby pomiarów wydajności obu algorytmów zmodyfikujemy nasze zbiory testowe 
opisane wyżej w następujący sposób.
\begin{enumerate}[a)]
    \item Losowe punkty $(x, y)$ w przestrzeni $\mathbb{R}^2$.
    \item Losowe punkty $(x, y)$ w przestrzeni $\mathbb{R}^2$, położone na okręgu.
    \item Losowe punkty w przestrzeni $\mathbb{R}^2$, 
    położone na prostokącie.
    \item Losowe punkty w przestrzeni $\mathbb{R}^2$, 
    położone na dolnym i lewym bokack kwadratu i jego przekątnych.
\end{enumerate}

\subsection{Wykresy zbiorów}
\begin{figure}[!h]
    \centering
    \begin{subfigure}{.5\textwidth}
      \centering
      \includegraphics[width=.9\linewidth]{a.png}
      \caption*{Rys. 1: Zbiór $a$.}
      \label{fig:sub1}
    \end{subfigure}%
    \begin{subfigure}{.5\textwidth}
      \centering
      \includegraphics[width=.9\linewidth]{b.png}
      \caption*{Rys. 2: Zbiór $b$.}
      \label{fig:sub2}
    \end{subfigure}
    \label{fig:test}
    \end{figure}
    \begin{figure}[!h]
    \centering
    \begin{minipage}{.5\textwidth}
      \centering
      \includegraphics[width=.9\linewidth]{c.png}
      \caption*{Rys. 3: Zbiór $c$.}
      \label{fig:test1}
    \end{minipage}%
    \begin{minipage}{.5\textwidth}
      \centering
      \includegraphics[width=.9\linewidth]{d.png}
      \caption*{Rys. 4: Zbiór $d$.}
      \label{fig:test2}
    \end{minipage}
    \end{figure}
\null
    \subsection{Algorytmy generacji zbiorów}
    \begin{enumerate}
    \item Dla zbioru a. Osobna generacja każdego z lososwych punktów.
    \item Dla zbioru b. Parametryzacja punktów na okręgu za pomocją funkcji trygonometrycznych $\sin$ i $\cos$.
    \item Dla zbioru c. Parametryzacja punktów na okręgu za pomocją funkcji trygonometrycznych $\sin$ i $\cos$.
    \item Dla zbioru d. Przekształcenie odcinka do postaci parametrycznej 
    $$
    l:\begin{cases}
      x = x_0 + tv_x& \smash{\raisebox{-1.6ex}{dla $\boldsymbol t \in [0,1]$.}}\\
      y = y_0 + tv_y
    \end{cases}
    $$
    A potem generowanie $t$ w podanym przedziale i dodanie odpowiednich punktów.
    \end{enumerate}
% \newpage

\section{Wydajność algorytmów}
\subsection{Opis zbiorów testowych}
\quad Na potrzeby pomiarów wydajności obu algorytmów zmodyfikujemy nasze zbiory testowe 
opisane wyżej w następujący sposób.
\begin{enumerate}[a)]
    \item Losowe punkty $(x, y)$ w przestrzeni $\mathbb{R}^2$, gdzie $(x, y) \in \left[-10^4,10^4\right]^{2}$.
    \item Losowe punkty $(x, y)$ w przestrzeni $\mathbb{R}^2$, położone na okręgu o promieniu $R = 10^4$.
    \item Losowe punkty w przestrzeni $\mathbb{R}^2$, 
    położone na prostokącie o wierzcholku w $(-10^4, -10^4)$ i $(10^4, 10^4)$.
    \item Losowe punkty w przestrzeni $\mathbb{R}^2$, 
    położone na dolnym i lewym bokack kwadratu o wierzcholku w $(-10^4, -10^4)$ i $(10^4, 10^4)$ i jego przekątnych.
\end{enumerate}

\quad Czasy wykonania algorytmów na zbiorach testowych w zależności od ilości punktów podane w sekundach $[s]$.
\renewcommand{\arraystretch}{2}
\begin{table}[!ht]
    \centering
\begin{tabular}{l  r| r| r| r|r|}
     & \multicolumn{5}{c}{\textbf{Liczba punktów}} \\ \cline{2-6}     
     \multicolumn{1}{l|}{\textbf{Algorytm}} & {100} & 1000 & 10000 & 100000 & 1000000 \\
    \hline
    \hline
    \multicolumn{1}{l|}{\textbf{Graham}} & 0.001&	0.007&	0.088	&1.222&	15.013 \\
    \hline
    \multicolumn{1}{l|}{\textbf{Jarvis}} & 0.001	&0.010	&0.107	&1.771	&25.266 \\
    \cline{2-6}
\end{tabular}
\caption*{Tabela 1: Czasy wykonania algorytmów na zbiorze testowym $a$.}
\end{table}

\begin{table}[ht]
    \centering
\begin{tabular}{l  r|r|r|r|}
    & \multicolumn{4}{c}{\textbf{Liczba punktów}} \\ \cline{2-5}     
    \multicolumn{1}{l|}{\textbf{Algorytm}} & 10& 100& 1000& 10000 \\
   \hline
   \hline
   \multicolumn{1}{l|}{\textbf{Graham}} & 0.0& 0.002& 0.023& 0.132 \\
   \hline
   \multicolumn{1}{l|}{\textbf{Jarvis}} & 0.0& 0.033& 0.727& 77.841 \\
   \cline{2-5}
\end{tabular}
\caption*{Tabela 2: Czasy wykonania algorytmów na zbiorze testowym $b$.}
\end{table}

\begin{table}[ht]
    \centering
\begin{tabular}{l  r|r|r|r|r|}
    & \multicolumn{5}{c}{\textbf{Liczba punktów}} \\ \cline{2-6}     
    \multicolumn{1}{l|}{\textbf{Algorytm}} &  10& 100& 1000& 10000& 100000 \\
   \hline
   \hline
   \multicolumn{1}{l|}{\textbf{Graham}} & 0.0& 0.004& 0.056& 0.759& 9.165 \\
   \hline
   \multicolumn{1}{l|}{\textbf{Jarvis}} & 0.0& 0.001& 0.011& 0.117& 1.254 \\
   \cline{2-6}
\end{tabular}
\caption*{Tabela 3: Czasy wykonania algorytmów na zbiorze testowym $c$.}
\end{table}

\begin{table}[!ht]
    \centering
\begin{tabular}{l  r|r|r|r|r|}
    & \multicolumn{5}{c}{\textbf{Liczba punktów}} \\ \cline{2-6}     
    \multicolumn{1}{l|}{\textbf{Algorytm}} &  100& 1000& 10000& 100000& 1000000 \\
   \hline
   \hline
   \multicolumn{1}{l|}{\textbf{Graham}} & 0.002& 0.027& 0.212& 2.816& 37.919 \\
   \hline
   \multicolumn{1}{l|}{\textbf{Jarvis}} & 0.001& 0.011& 0.077& 0.921& 9.709 \\
   \cline{2-6}
\end{tabular}
\caption*{Tabela 4: Czasy wykonania algorytmów na zbiorze testowym $d$.}
\end{table}
\section{Omówienie otrzymanych wyników}
\quad Krótko omówimy dlaczego zostały wybrane takie zbiory do testowania algorytmów oraz jakie problemy były tym spowodowane.
\par
Zbiór $a$ to po prostu zbiór losowych, więc w naszy przypadku jest przydatny, jako weryfikujący działanie algorytmów w przypadku najbardziej ogólnym. 
Zbiór $b$ (okrąg) jest zbiorem, który stwarza najwięcej problemów dla algorytmu Jarvisa (dokładniej omówione poniżej) i potrzebny dla demonstracji różnicy pomiędzy dwoma algorytmami w 
przypadku skrajnym. Zbiory $c$ i $d$ charakteryzują się małą otoczką wypukłą oraz dużą ilością punktów współliniowych. To jest dobrą demostracją prtzypadku, dla którego algorytm Jarvisa jest bardziej wydajny, 
oraz testuje naszą możliwość wyznaczania punktów współliniowych w algorytmie Grahama.
\par Omówimy teraz otrzymane wyniki czasów wykonania obu algorytmów dla zbiorów testowych. 
\begin{enumerate}[a)]
    \item Dla zbiorów lososwych w których ilość punktów była mniejsza od $10^4$ algorytm Grahama i Jarvisa 
    mają prawie identyczne czasy wykonania. Dla zbiorów o większej mocy Graham mający mniejszą złożoność, jest 
    zdecydowanie szybszy, co nie powinno nas dziwić. 
    \item Dla punktów na okręgu widzimy jeszcze wiekszą przewagę algorytmu Grahama niż z poprzednim zbiorem. 
    Spowodowane to jest tym, że w okręgu każdy z punktów należy do otoczki wypukłej, a złożoność Jarvisa $O(nh)$, zależy od 
    ilości punktów w otoczce $h$. Dlatego w pewnym sensie to jest najgorszy przypadek dla algorytmu Jarvisa i jego złożoność staje się $O(n^2)$ w porównaniu do złożoności Grahama $O(nlog(n))$ .
    \item Sytuację odwrotną widzimy dla prostokąta. Jego otoczka składa się tylko z 4 punktów (wierzchołków). 
    i w  tym przypadku Jarvis faktycznie ma złożoność liniową, a Graham nadal pozostaje ze złożonością liniowo-logarytmiczną, co wyraźnie wydać po wynikach. 
    \item Dla kwadratu i jego przekątnych mamy podobną sytuację, jak i w przypadku z prostokątem. Otoczka wypukła tego zbioru punktów składa się 
    z małej ilości punktów w porównaniu do mocy całego zbioru. To powoduje, że Jarvis jest znacznie bardziej wydajny i llepiej radzi sobie z problemem. 
    Graham zaczynając od lewego dolnego wierzchołka $p_0$ sortuje wszystkie punkty względem kątów nachylenia prostych do nich prowadzących od $p_0$, to też może powodować niektóre problemy, gdzyż w tym przypadku, jak i w zbiorze $c$, mamy bardzo dużo punktów współliniowych i ich poprawne posortowanie dla zbiorów na dużych zakresach nie jest takie łatwe ze względu na możliwe błędy spowodowane precyzją obliczeń arytmetycznych.
\end{enumerate}

\section{Podsumowanie}
\quad W trakcie prowadzenia tego badania przeprowadzona została analiza i porównanie dwóch popularnych algorytmów służących do wyznaczania otoczki wypukłej, tj. algorytmów Grahama i Jarvisa. Celem było zrozumienie działania tych algorytmów oraz porównanie ich wydajności przy użyciu różnych zbiorów punktów. 
Poniżej przedstawione są główne wnioski z  badań.
\par
Algorytm Grahama charakteryzuje się złożonością czasową wynoszącą $O(nlog(n))$, co sprawia, że jest bardziej wydajny niż algorytm Jarvisa dla większości zbiorów losowych, azckolwiek są 
wyjątki co widać po zbiorach $c$ i $d$. W związku ze swoją lepszą złożonością używanie algorytma Grahama jest zalecane dla większości przypadków.

\par
Algorytm Jarvisa jest prosty w implementacji, ale jego złożoność czasowa wynosi $O(nh)$, gdzie $n$ to liczba punktów, a $h$ to liczba punktów na otoczce wypukłej.
Jego wydajność maleje znacząco w przypadku zbiorów z otoczką wypukłą o dużym rozmiarze, co sprawia, że może być mniej praktyczny w niektórych sytuacjach.
Wykorzystanie algorytmu Jarvisa może być zalecane tylko w przypadkach gdy wiemy, że otoczka wypukła składa się ze stosunkowo niedużej ilosci punktów.

Zaobserwowaliśmy, że algorytm Grahama osiąga lepsze wyniki wydajnościowe niż algorytm Jarvisa w przypadku dużych zbiorów punktów losowych. 
Natomiast jeżeli nasz zbiór punktów ma pewną strukturę, która gwarantuje stosunkowo małą otoczkę wypukłą, to użycie algorytmu Jarvisa może być troche lepsze dla przypadków o dużej ilości punktów.
Algorytm Grahama jest bardziej efektywny dzięki zastosowaniu sortowania punktów względem kąta względem punktu startowego, co pozwala na znaczne przyspieszenie procesu wyznaczania otoczki wypukłej.

Podsumowując, zarówno algorytm Grahama, jak i algorytm Jarvisa znajdują zastosowanie w wyznaczaniu otoczki wypukłej, ale wybór między nimi zależy od konkretnych potrzeb i rozmiaru zbioru punktów. Algorytm Grahama jest bardziej efektywny i zwykle bardziej polecany w praktyce, zwłaszcza w przypadku dużych zbiorów punktów, gdzie złożoność czasowa odgrywa kluczową rolę. Jednak algorytm Jarvisa pozostaje przydatnym narzędziem, zwłaszcza w celach edukacyjnych i w przypadku mniejszych zbiorów punktów, gdzie prostota implementacji może mieć znaczenie.
\end{document}