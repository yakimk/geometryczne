\documentclass{sprawozdanie-agh}

\usepackage[utf8]{inputenc}
\usepackage{listings}
\usepackage{tikz}
\usepackage{xcolor}
\usepackage{graphicx}
\usepackage{caption}
\usepackage{graphicx}
\usepackage{lipsum}
\usepackage{wrapfig}
\usepackage{subcaption}
\usepackage{accsupp}
\usepackage{array}
\usepackage{multirow}
\usepackage{amsmath, siunitx}
\usepackage{booktabs}
\graphicspath{ {./images/} }

\usetikzlibrary{angles,quotes}
\makeatletter

\begin{document}

\przedmiot{Algorytmy geometryczne}
\tytul{Laboratorium 2}
\podtytul{Otoczka wypukła}
\kierunek{Informatyka}
\autor{Kyrylo Iakymenko}
\data{Kraków, 7 listopada 2023}

\stronatytulowa{}

\section{Wprowadzenie}
\quad Czasy wykonania algorytmów na zbiorach testowych w zależności od ilości punktów podane w sekundach $[s]$.
\renewcommand{\arraystretch}{2}
\begin{table}[!ht]
    \centering
\begin{tabular}{l  r| r| r| r|r|}
     & \multicolumn{5}{c}{\textbf{Liczba punktów}} \\ \cline{2-6}     
     \multicolumn{1}{l|}{\textbf{Algorytm}} & {100} & 1000 & 10000 & 100000 & 1000000 \\
    \hline
    \hline
    \multicolumn{1}{l|}{\textbf{Graham}} & 0.001&	0.007&	0.088	&1.222&	15.013 \\
    \hline
    \multicolumn{1}{l|}{\textbf{Jarvis}} & 0.001	&0.010	&0.107	&1.771	&25.266 \\
    \cline{2-6}
\end{tabular}
\caption*{Tabela 1: Czasy wykonania algorytmów na zbiorze testowym $a$.}
\end{table}

\begin{table}[ht]
    \centering
\begin{tabular}{l  r|r|r|r|}
    & \multicolumn{4}{c}{\textbf{Liczba punktów}} \\ \cline{2-5}     
    \multicolumn{1}{l|}{\textbf{Algorytm}} & 10& 100& 1000& 10000 \\
   \hline
   \hline
   \multicolumn{1}{l|}{\textbf{Graham}} & 0.0& 0.002& 0.023& 0.132 \\
   \hline
   \multicolumn{1}{l|}{\textbf{Jarvis}} & 0.0& 0.033& 0.727& 77.841 \\
   \cline{2-5}
\end{tabular}
\caption*{Tabela 2: Czasy wykonania algorytmów na zbiorze testowym $b$.}
\end{table}

\begin{table}[ht]
    \centering
\begin{tabular}{l  r|r|r|r|r|}
    & \multicolumn{5}{c}{\textbf{Liczba punktów}} \\ \cline{2-6}     
    \multicolumn{1}{l|}{\textbf{Algorytm}} &  10& 100& 1000& 10000& 100000 \\
   \hline
   \hline
   \multicolumn{1}{l|}{\textbf{Graham}} & 0.0& 0.004& 0.056& 0.759& 9.165 \\
   \hline
   \multicolumn{1}{l|}{\textbf{Jarvis}} & 0.0& 0.001& 0.011& 0.117& 1.254 \\
   \cline{2-6}
\end{tabular}
\caption*{Tabela 3: Czasy wykonania algorytmów na zbiorze testowym $c$.}
\end{table}

\begin{table}[!ht]
    \centering
\begin{tabular}{l  r|r|r|r|r|}
    & \multicolumn{5}{c}{\textbf{Liczba punktów}} \\ \cline{2-6}     
    \multicolumn{1}{l|}{\textbf{Algorytm}} &  100& 1000& 10000& 100000& 1000000 \\
   \hline
   \hline
   \multicolumn{1}{l|}{\textbf{Graham}} & 0.002& 0.027& 0.212& 2.816& 37.919 \\
   \hline
   \multicolumn{1}{l|}{\textbf{Jarvis}} & 0.001& 0.011& 0.077& 0.921& 9.709 \\
   \cline{2-6}
\end{tabular}
\caption*{Tabela 4: Czasy wykonania algorytmów na zbiorze testowym $d$.}
\end{table}

\section{Opis stanowiska}

\section{Przebieg doświadczenia}


\section{Wyniki pomiarów}


\section{Opracowanie wyników}
\section{Podsumowanie}
\end{document}
