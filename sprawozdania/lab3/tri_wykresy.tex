\begin{figure}[!h]
    \centering
    \begin{subfigure}{.5\textwidth}
      \centering
      \includegraphics[width=.9\linewidth]{polygon_5-tri.png}
      \caption*{Rys. 7: Triangulacja pięciokąta.}
      \label{fig:sub1}
    \end{subfigure}%
    \begin{subfigure}{.5\textwidth}
      \centering
      \includegraphics[width=.9\linewidth]{christmas_tree-tri.png}
      \caption*{Rys. 8: Triangulacja choinki.}
      \label{fig:sub2}
    \end{subfigure}
    \label{fig:test}
    \end{figure}

\newpage

    \begin{figure}[!h]
    \centering
    \begin{minipage}{.5\textwidth}
      \centering
      \includegraphics[width=.9\linewidth]{polygon_weird-tri.png}
      \caption*{Rys. 9: Triangulacja wielokąta 1.}
      \label{fig:test1}
    \end{minipage}%
    \begin{minipage}{.5\textwidth}
      \centering
      \includegraphics[width=.9\linewidth]{polygon_weird_2-tri.png}
      \caption*{Rys. 10: Triangulacja wielokąta 2.}
      \label{fig:test2}
    \end{minipage}
    \end{figure}
    \begin{figure}[!h]
        \centering
        \begin{minipage}{.5\textwidth}
          \centering
          \includegraphics[width=.9\linewidth]{polygon_weird_3-tri.png}
          \caption*{Rys. 11: Triangulacja wielokąta 3.}
          \label{fig:test1}
        \end{minipage}%
        \begin{minipage}{.5\textwidth}
          \centering
          \includegraphics[width=.9\linewidth]{polygon_50-tri.png}
          \caption*{Rys. 12: Triangulacja 50-kąta foremnego.}
          \label{fig:test2}
        \end{minipage}
        \end{figure}